\documentclass[12pt, a4paper]{article}
\usepackage{fullpage}
\usepackage{amsmath}
\usepackage{graphicx}
\usepackage{natbib}
%\usepackage{epstopdf}
%\usepackage{chemstyle}
\title{Geochemical modeling of CH$_4$ and CO$_2$ production from anaerobic Arctic soil incubation microcosms}
\author{Guoping Tang}
\date\today{}

\begin{document}
\maketitle

\begin{abstract}
\end{abstract}

\begin{figure}
\centering
\includegraphics[width=0.6\textwidth]{fig1/fig1.pdf}
\caption{Extend the CLM-CN decomposition cascade (Thornton and Rosenbloom,
2005) to include a LabileDOC pool. A fraction of the original respiration fraction is assumed to
produce LabileDOC, which undergoes fermentation, Fe reduction and
methanogenesis to release CO$_2$ and CH$_4$. FeRB, MeGA, and MeGH denote microbial
mass pools for Fe reducers, acetaclastic and hydrogenotrophic methanogens,
respectively. $\tau$ is the turnover time.}
\label{fig1}
\end{figure}

\begin{figure}
\centering
\includegraphics[width=1.0\textwidth]{fig2/fig2.pdf}
\caption{
Observed CO$_2$ (a1-6) and CH$_4$ (b1-6) in the head space, and organic acids (Ac, c1-6),
extractable Fe(II) (d1-6), and pH (e1-6) in the incubation tests
with soils from an Arctic lower center polygon. Symbols represent
observations, with blue, green and red for -2, 4, and 8 $^{\circ}$C. For CO$_2$ and CH$_4$, different symbols of the same color represent
duplicates. The organic acids, such formate, acetate, propionate, and butyrate,
reported by (Herndon et al., 2015)  are lumped as Ac in c1-6. The rest of the
data were taken from (Roy Chowdhury et al., 2015). The curves are model
calculations based on model parameter values listed in Table 1 and experimental
parameter values listed in Table 2. Trough, ridge, and center denote the
microtopographic locations in the polygon, and mineral and organic denote soil
horizons.}
\label{fig2}
\end{figure}

\begin{figure}
\centering
\includegraphics[width=1.0\textwidth]{fig3/fig3.pdf}
\caption{
Increasing initial bioavailable Fe(III) gets predictions close to observations for Fe(II) and pH for center and ridge organic soils. Refer to Fig. 2 for more information.
}
\label{fig3}
\end{figure}

\begin{figure}
\centering
\includegraphics[width=0.6\textwidth]{fig4/fig4.pdf}
\caption{
Partition of CO$_2$ among gas and aqueous phase species under various
temperatures. The calculations are conducted with 45 ml head space with N$_2$ and
10 ml solution with 10 mM total inorganic carbon using PHREEQC. 
}
\label{fig4}
\end{figure}

\begin{figure}
\centering
\includegraphics[width=0.6\textwidth]{fig5/fig5.pdf}
\caption{
Increasing the initial LabileDOC better describes the observed initial rapid CO$_2$
increase in the headspace for the organic soils. 
}
\label{fig5}
\end{figure}

\begin{figure}
\centering
\includegraphics[width=0.6\textwidth]{fig6/fig6.pdf}
\caption{
Increasing the initial biomass predicts fast CH$_4$ accumulation at early times
that is close to the observations but misses the level-off trend at late time
for the center organic soils.
}
\label{fig6}
\end{figure}

\begin{figure}
\centering
\includegraphics[width=0.6\textwidth]{fig7/fig7.pdf}
\caption{
Comparison of pH response functions used in CLM4Me (Riley et al., 2011), TEM
(Raich et al. 1991), and DLEM (Tian et al., 2010) as described
by Eqs. 3, A1-3.
}
\label{fig7}
\end{figure}

\begin{figure}
\centering
\includegraphics[width=1.0\textwidth]{fig8/fig8.pdf}
\caption{
Comparison of temperature response functions used in land surface models CLM-CN
(Thornton and Rosenbloom, 2005), CENTURY (Parton et al., 2010), Q10 (Oleson et
al., 2013), Arrhenius equation (Wang et al. 2013), and Ratkowsky equation  (Ratkowsky et al.,
1982) described by Eq. (4, B1-B4). 
}
\label{fig8}
\end{figure}

\begin{figure}
\centering
\includegraphics[width=1.0\textwidth]{figs1/figs1.pdf}
\caption{
Calculated partition of CO$_2$ in gas and aqueous phases as a percentage of
initial TOTC with different $f_{Fe3}$ values. The results correspond to Fig. 3 for
temperature 8 $^{\circ}$C.  With increasing $f_{Fe3}$, the pH increases at the late times, so
does the CO$_2$ solubility.  
}
\label{figs1}
\end{figure}

\begin{figure}
\centering
\includegraphics[width=0.6\textwidth]{figs2/figs2.pdf}
\caption{
Adding 1 mmol Fe(OH)$_{3a}$ into the numerical experiments shown in Fig. 4, the gas
phase fraction is decreased at low pH values.
}
\label{figs2}
\end{figure}

\begin{figure}
\centering
\includegraphics[width=1.0\textwidth]{figs3/figs3.pdf}
\caption{
Impact of adsorption of CO2 to ferric oxide surfaces on the distribution among gas, aqueous and solid phases.
\label{figs3}
\end{figure}

\begin{figure}
\centering
\includegraphics[width=1.0\textwidth]{figs4/figs4.pdf}
\caption{
Partition of carbon among various organic pools.
a, b, and c are for CO$_2$ distribution in the gas (head space),
aqueous (water), and adsorbed (sorption to Fe(OH)$_{3a}$).  See Fig.2 caption for
more description about the model and experimental parameters.
}
\label{figs4}
\end{figure}

\begin{figure}
\centering
\includegraphics[width=1.0\textwidth]{figs5/figs5.pdf}
\caption{
Impact of specified organic matter content in WHAM on predictions. See Fig.2 caption for more
description about the model and experimental parameters.
}
\label{figs5}
\end{figure}

\begin{figure}
\centering
\includegraphics[width=1.0\textwidth]{figs6/figs6.pdf}
\caption{
Comparison of impact of different pH response functions (CLM4Me, TEM, and DLEM) on predictions.
}
\label{figs6}
\end{figure}

\begin{figure}
\centering
\includegraphics[width=1.0\textwidth]{figs7/figs7.pdf}
\caption{
Fig. 8 with arithmetic vertical scale.
}
\label{figs7}
\end{figure}

\begin{figure}
\centering
\includegraphics[width=1.0\textwidth]{figs8/figs8.pdf}
\caption{
Comparison of impact of different temperature response functions (CLM-CN, CENTURY, Ratkowsky equation with $T_m$ =260) on predictions.
}
\label{figs8}
\end{figure}

\begin{figure}
\centering
\includegraphics[width=1.0\textwidth]{figs9/figs9.pdf}
\caption{
Comparison of impact of different temperature response functions (CLM-CN, Arrhenius equation ($E_a$), Q$_{10}$ equation)
}
\label{figs9}
\end{figure}

#\clearpage
%\bibliographystyle{plain}
%#\bibliographystyle{plainnat}
%\bibliography{monod}

\end{document}
