\documentclass[12pt, a4paper]{article}
\usepackage{fullpage}
\usepackage{amsmath}
\usepackage{graphicx}
\usepackage{natbib}
%\usepackage{epstopdf}
%\usepackage{chemstyle}
\title{Geochemical modeling of CH$_4$ and CO$_2$ production from anaerobic Arctic soil incubation microcosms}
\author{Guoping Tang}
\date\today{}

\begin{document}
\maketitle

\begin{abstract}
\end{abstract}

\begin{figure}
\centering
\includegraphics[width=0.6\textwidth]{fig1/fig1.pdf}
\caption{Extend the CLM-CN decomposition cascade (Thornton and Rosenbloom,
2005) to include production of a LabileDOC pool, anaerobic Fe reduction and
methanogenesis. A fraction of the original respiration fraction is assumed to
produce LabileDOC, which undergoes fermentation, Fe reduction and
methanogenesis to release CO2 and CH4. FeRB, MeGA, and MeGH denote microbial
mass pools for Fe reducers, acetaclastic and hydrogenotrophic methanogens,
respectively. $\tau$ is the turnover time.}
\label{fig1}
\end{figure}

\begin{figure}
\centering
\includegraphics[width=1.0\textwidth]{fig2/fig2.pdf}
\caption{
Observed CO$_2$ (a1-6) and CH4 (b1-6) in the head space, and acetate acid (c1-6),
extractable Fe(II) (d1-6), and pH (e1-6) in the soils from incubation tests
with a soils from an Arctic  lower center polygon. Symbols represent
observations, with blue, green and red for -2, 4, and 8 °C incubation
temperatures. For CO$_2$ and CH$_4$, different symbols of the same color represent
duplicates. The organic acids, such formate, acetate, propionate, and butyrate,
reported by (Herndon et al., 2015)  are lumped as Ac in c1-6. The rest of the
data were taken from (Roy Chowdhury et al., 2015). The curves are model
calculations based on model parameter values listed in Table 1 and experimental
parameter values listed in Table 2. Trough, ridge, and center denote the
topographic locations in the polygon, and mineral and organic denote soil
horizons.}
\label{fig2}
\end{figure}

\begin{figure}
\centering
\includegraphics[width=1.0\textwidth]{fig3/fig3.pdf}
\caption{
Added predictions with increasing initial bioavailable Fe(III) to Fig. 2.
}
\label{fig3}
\end{figure}

\begin{figure}
\centering
\includegraphics[width=0.6\textwidth]{fig4/fig4.pdf}
\caption{
Partition of CO$_2$ among gas and aqueous phase species under various
temperatures. The calculations are conducted with 45 ml head space with N$_2$ and
10 ml solution with 10 mM TIC using PHREEQC. 
}
\label{fig4}
\end{figure}

\begin{figure}
\centering
\includegraphics[width=0.6\textwidth]{fig5/fig5.pdf}
\caption{
Increasing the LabileDOC pool better describes the observed initial rapid CO$_2$
increase in the headspace for the organic soils. 
}
\label{fig5}
\end{figure}

\begin{figure}
\centering
\includegraphics[width=0.6\textwidth]{fig6/fig6.pdf}
\caption{
Increasing the initial biomass predicts fast CH$_4$ accumulation at early times
that is close to the observations but misses the level-off trend at late time
for the organic soils. It is not clear what limited methanogenesis as
substantial acetate remained. 
}
\label{fig6}
\end{figure}

\begin{figure}
\centering
\includegraphics[width=0.6\textwidth]{fig7/fig7.pdf}
\caption{
Comparison of pH response functions used in CLM4Me (Riley et al., 2011), TEM
(Cao et al., 1995;Xu et al., 2015), and DLEM (Tian et al., 2010) as described
by Eqs. 3, A1-3.
}
\label{fig7}
\end{figure}

\begin{figure}
\centering
\includegraphics[width=1.0\textwidth]{fig8/fig8.pdf}
\caption{
Comparison of temperature response functions used in land surface models CLM-CN
(Thornton and Rosenbloom, 2005), CENTURY (Parton et al., 2010), Q10 (Oleson et
al., 2013), Arrhenius equation , and Ratkowsky equation  (Ratkowsky et al.,
1982) described by Eq. (4, B1-B4). 
}
\label{fig8}
\end{figure}

\begin{figure}
\centering
\includegraphics[width=1.0\textwidth]{figs1/figs1.pdf}
\caption{
Calculated partition of CO2 in gas and aqueous phases as a percentage of
initial TOTC with different fFe3 values. The results correspond to Fig. 3 for
temperature 8 °C.  With increasing fFe3, the pH increases at the late times, so
does the solubility in trough and center mineral soils.  
}
\label{figs1}
\end{figure}

\begin{figure}
\centering
\includegraphics[width=0.6\textwidth]{figs2/figs2.pdf}
\caption{
Adding 1 mmol Fe(OH)3a into the numerical experiments shown in Fig. 6, the gas
phase fraction is decreased at low pH values.
}
\label{figs2}
\end{figure}

\begin{figure}
\centering
\includegraphics[width=1.0\textwidth]{figs3/figs3.pdf}
\caption{
Assuming more initial SOM1 fraction (double) for the organic soils and less (half) for the mineral soils, the model better describes the observed more CO2 head space concentrations in the organic soils than in the mineral soils. The dash curves are identical to Fig. 2 for reference. See Fig. 2 caption for more description.}
\label{figs3}
\end{figure}

\begin{figure}
\centering
\includegraphics[width=1.0\textwidth]{figs4/figs4.pdf}
\caption{
Partition of carbon among various organic pools for the base case simulations
shown in Fig. 2. a, b, and c are for CO2 distribution in the gas (head space),
aqueous (water), and adsorbed (sorption to Fe(OH)3a).  See Fig.2 caption for
more description about the model and experimental parameters.
}
\label{figs4}
\end{figure}

\begin{figure}
\centering
\includegraphics[width=1.0\textwidth]{figs5/figs5.pdf}
\caption{
Specified organic matter in WHAM on predictions. See Fig.2 caption for more
description about the model and experimental parameters.
}
\label{figs5}
\end{figure}

\begin{figure}
\centering
\includegraphics[width=1.0\textwidth]{figs6/figs6.pdf}
\caption{
Comparison of different pH response functions (CLM4Me, TEM, and DLEM).
}
\label{figs6}
\end{figure}

\begin{figure}
\centering
\includegraphics[width=1.0\textwidth]{figs7/figs7.pdf}
\caption{
Fig. 8 with arithmetic vertical scale. See Fig. 3 caption for more description.
}
\label{figs7}
\end{figure}

\begin{figure}
\centering
\includegraphics[width=1.0\textwidth]{figs8/figs8.pdf}
\caption{
Comparison of different temperature response functions (CLM-CN, CENTURY, Ratkowsky Equation with Tm =260).
}
\label{figs8}
\end{figure}

\begin{figure}
\centering
\includegraphics[width=1.0\textwidth]{figs9/figs9.pdf}
\caption{
Comparison of different temperature response functions (CLM-CN, Arrhenius equation (Ea), Q10 Equation)
}
\label{figs9}
\end{figure}











\clearpage
%\bibliographystyle{plain}
%#\bibliographystyle{plainnat}
%\bibliography{monod}

\end{document}
